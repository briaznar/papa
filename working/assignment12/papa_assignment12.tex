% Enable warnings about problematic code
\RequirePackage[l2tabu, orthodox]{nag}

\documentclass{WeSTassignment}

% The lecture title, e.g. "Web Information Retrieval".
\lecture{Introduction to Web Science}
% The names of the lecturer and the instructor(s)
\author{%
  Prof. Dr.~Steffen~Staab\\{\normalsize\mailto{staab@uni-koblenz.de}} \and
  Ren{\'e}~Pickhardt\\{\normalsize\mailto{rpickhardt@uni-koblenz.de}} \and
   Korok~Sengupta\\{\normalsize\mailto{koroksengupta@uni-koblenz.de}} \and 
   Olga~Zagovora\\{\normalsize\mailto{zagovora@uni-koblenz.de}}
}
% Assignment number.
\assignmentnumber{12}
% Institute of lecture.
\institute{%
  Institute of Web Science and Technologies\\%
  Department of Computer Science\\%
  University of Koblenz-Landau%
}
% Date until students should submit their solutions.
\datesubmission{February 15, 2016, 10:00 a.m.}
% Date on which the assignments will be discussed in the tutorial.
\datetutorial{February 17, 2016, 12:00 p.m.}

% Set language of text.
\setdefaultlanguage[
  variant = american, % Use American instead of British English.
]{english}

% Specify bib file location.
\addbibresource{bibliography.bib}

% For left aligned centered boxes
% see http://tex.stackexchange.com/a/25591/75225
\usepackage{varwidth}

% ==============================================================================
% Document

\begin{document}

\maketitle
This assignment is about \textbf{Net Neutrality \& Copyright}

Copying answers straight way from any source wont be considered for the final score of this assignment. Please cite your sources if any.\\ \\ 

%Please mention your team Names here: 
Team Name: papa
\\Members:
\\Brigitte Aznar
\\Bonasmitha Behura
\\Ilia Tugushi


\section{Copyright \& Creative Commons (10 points)}

\subsection{Differences}
On what grounds can you differentiate between Copyright and Creative Commons?\\

\textbf{Answer:}
\\According to Wikipedia Copyright is "a legal right created by the law of a country that grants the creator of an original work exclusive rights for its use and distribution."
On the other hand Creative commons is a license the author of a certain content chooses to grant others. Which it is part of copyright but what differentiates them is that the later is not the law itself but a permission given by the author to use the their content.
\subsubsection{Case Study}
Let us consider that Donald has an idea to develop a study material for the poor  children from his area who cannot attend a school. But in order to have this idea as a product, he needs some financial help from investors so that he can collect materials and also to set up a website where kids can study for free using the materials and videos he makes for them. But Donald wants it to be completely free and shareable so that it can help anyone. 

\begin{itemize}
\item Can Donald's \emph{idea} be copyrighted?
\item How can Donald protect his idea when he presents it to the investors?
\item Since the investors are investing capital, can they still recover money if Donald wants to go for the Creative Common licenses? If so, state the ways?  \\
\end{itemize}

\textbf{Answer:}
\begin{itemize}
\item It can be copyrighted, but since he wants it to be for free for everybody perhaps Donald should use a Creative Commons license.
\item He might specify a clause under the license saying that all materials are free to use and distribute under his platform. So no new person or entity can redistribute it with financial means.
\item The investors and Donald can come up with a different form of financing, the project which is unrelated to the content and videos he postes. for example advertising or maybe charging the government for it.
\end{itemize}

%-------------------------------------------------------------------------------

\section{Neutrality(10 points)}

\begin{itemize}
\item Define the term \emph{net neutrality}.
\item Argue for and against the motion on the concept of priority pricing as discussed in Kögler et.al(2011)\footnote{\label{note1}Berger-Kögler, U. and Kruse, J. (2011) ‘Net neutrality regulation of the internet?’, Int. J. Management and Network Economics, Vol. 2, No. 1, pp.3–23.}
\item - Explain why?\quote{"...additional internet capacity would not lead to additional revenues because of the flat rates."}\textsuperscript{\ref{note1}}\\

\end{itemize}

\textbf{Answers:}
\begin{itemize}
\item Net Neutrality refers to the principle that governments and/or ISPs are not allowed to censor or block access to certain internet content.
\item In the paper it is mentioned that priority pricing is like a pre agreement of a person with their ISP to have preferential access to the internet in case the network is oveloaded (at a higher price), Which means their Internet access wouldn't decrease (at least not drastically)
\begin{itemize}
\item \textbf{For}: This is a good method if the client is for instance a company treating important data that must always be accessible (disregarding having ISP redundancy) who wishes a guarantee to always 24/7 no matter what have internet access, and they pay extra for this security.
\item  \textbf{Against}: This translates to higher fee from the ISP, and can ultimately translate in a higher price for the users of certain online service, but we believe this would probably be a small price to pay for them in exchange of around the clock access to their services.
\end{itemize}
\item It is the same principle as when a product is produced in a factory, if the factory only fabricates a handful of product X the costs will be higher, but if the factory mass produces X then they can only optimize costs so high, when they reach this optimal point which is when costs are low and revenue is high, even if they produce more of X the costs will remain the same.
\end{itemize}


\makefooter

\end{document}
